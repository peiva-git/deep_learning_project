%! Author = Federico Calandra, Ivan Pelizon
%! Date = 24/10/23

\section{Abstract}\label{sec:abstract}

    Image denoising is a fundamental problem in computer vision, with a wide range of real-world applications,
    including improving the quality of images for object segmentation, detection, tracking, and more.
    This project aims to explore and implement state-of-the-art deep learning models for image denoising,
    with a particular focus on CBDNet, a convolutional blind denoising network.

    Deep Learning has demonstrated remarkable success in this field~\cite{9057895},
    and our project would leverage this technology to address the issue of noisy images.
    We’d delve into the principles of deep learning, especially Convolutional Neural Networks (CNNs),
    and understand how these networks can be trained to remove noise and enhance image quality.

    One of the key highlights of this project is the exploration of CBDNet, a novel approach to image denoising~\cite{guo2019convolutional}.
    CBDNet focuses on improving the generalization ability of deep CNN denoisers by training with realistic noise models and real-world noisy-clean image pairs.
    It considers signal-dependent noise and in-camera signal processing pipelines to synthesize realistic noisy images.
    Additionally, a noise estimation subnetwork is embedded within CBDNet to rectify denoising results conveniently.

    We would implement both CBDNet and simpler deep learning models for image denoising using Python.
    After this stage, we would perform an extensive analysis of the results obtained.
    Evaluation will include qualitative and quantitative metrics.
    Qualitatively, aspects such as edge preservation, texture, uniformity, and smoothness will be considered.
    Quantitative metrics like Peak Signal-to-Noise Ratio (PSNR)~\cite{NI_official_website_2023},
    Structural Similarity Index Measurement (SSIM)~\cite{1284395},
    and Mean Square Error (MSE) will be used to compare the performance of the deep learning-based denoising methods with traditional techniques.
    The project will provide a clear understanding of the strengths and limitations of different models.
